\begin{abstractFrame}
\begin{deAbstract}
Diese Arbeit befasst sich mit der Überprüfung von Graphen, mit dem Ziel in 
computerbasierten Modellierungsaufgaben die Eingabe eines Lerners mit einer 
Musterlösung zu vergleichen. 
Dazu wird das Prinzip des Graphabstands theoretisch betrachtet 
und Algorithmen vorgestellt, die diesen ermitteln. Die Algorithmen werden anschließend 
getestet, wobei die Laufzeit im Vordergrund steht. Dabei brachten ein 
Branch-and-Bound-Verfahren sowie ein evolutionärer Algorithmus die besten Ergebnisse.
 Zusätzlich wird ein Konzept 
vorgestellt, mit dem die Eingabe eines Lerners genauer ausgewertet werden kann.
\end{deAbstract}

%\vfil

\begin{enAbstract}
This thesis deals with the checking of graphs with the goal of comparing the input of 
a student with a reference solution for a computerbased modelling task. Therefore the 
principle of graph edit distance 
is threated theoretically and algorithms are be introduced to compute graph edit 
distance. Next the algorithms will be tested with the focus on runtime. In the tests a 
branch-and-bound-procedure and an evolutionary algorithm yielded the best results. Also this 
thesis introduces a concept that allows analysing the input of the student.
\end{enAbstract}

\end{abstractFrame}
