\chapter{Diskussion}

Die Arbeit beschäftigte sich mit dem Vergleichen von Graphen. Ziel war es 
eine algorithmische Grundlage zu schaffen, um die Eingabe eines Lerners 
bewerten zu können. 

\section{Die untersuchten Algorithmen}
Um das Ziel zu erreichen wurden verschiedene Algorithmen betrachtet. 
Einige erwiesen sich als gute Grundlage für weitere Betrachtungen. Andere 
hingegen scheiterten bereits an kleinen Graphenpaaren.


\subsection{Kantengraphen und MCS}
Ein Ansatz zum Ermitteln einer Lösung war es, die zu vergleichenden 
Graphen in Kantengraphen umzuwandeln. Als nächstes sollte dann der 
gemeinsame induzierte Teilgraph (MCS) ermittelt werden. Dieser Ansatz 
erwies sich jedoch als unbrauchbar. Die Laufzeit ist schlicht zu hoch. Die 
dafür getesteten Verfahren waren jedoch so ausgelegt, dass sie alle 
Möglichkeiten durchprobieren. Wäre es möglich die Laufzeit auf ein 
vertretbares Maß zu senken, dann ließe sich auch untersuchen, ob der 
Ansatz eines Kantengraphen sinnvoll ist. Dieser bietet die Möglichkeit 
Kanten unabhängig von Knoten zu betrachten. Es besteht die Hoffnung, dass 
man auf diese Weise zusätzliche Informationen erhält. Eventuell lassen 
sich aber die gleichen Informationen auch über Mustersuche im Fehlergraphen 
ermitteln.


\subsection{Probieren aller Permutationen}
Als nächstes wurde versucht ECGMs direkt zu ermitteln. Der erste Ansatz 
dabei war das Probieren aller möglichen Permutationen. Das Verfahren 
ist zwar offensichtlich eines, welches sehr schnell eine zu hohe 
Laufzeit besitzt, jedoch war es deutlich schneller als die Suche nach 
einem MCS in Kantengraphen für die gleichen Graphenpaare. Es ist somit 
als Vergleichsgröße zu verstehen.


\subsection{Das B\&B-Verfahren}
Das B\&B-Verfahren erwies sich als ein zufriedenstellendes. Die Idee daran, 
abzuschätzen wo im Suchbaum sich das Optimum befindet, beschleunigt die 
Suche stark. Dabei wurde beiläufig festgestellt, dass es sich bei B\&B sowie 
dem A*-Algorithmus um fast identische Verfahren handelt. Dies ist insofern 
interessant, weil diese Ähnlichkeit üblicherweise in entsprechender 
Literatur nicht erwähnt wird. Ursache dafür mag sein, dass die Intention, 
die zu diesen Algorithmen führte, bei beiden eine unterschiedliche war. 
Bei B\&B ging es um die Lösung linearer Optimierungsprobleme. Der 
A*-Algorithmus hat seine Wurzeln in der Pfadsuche.


\subsection{Verbesserung von B\&B}
Ebenfalls interessant ist die starke Streuung bei der Laufzeit des 
B\&B-Verfahrens. Hier bleibt die Frage offen, wie sich die Streuung und 
die Laufzeit verringern lassen. Dabei bieten sich zwei Ansätze: das 
Finden einer besseren Schranke und das Erkennen und Nutzen von 
Eigenschaften, die sich auf den Suchaufwand auswirken.


\subsubsection{Eine bessere Schranke}
Je besser die Schranke ist, umso weniger Varianten müssen überprüft 
werden. Dabei gilt es jedoch zu beachten, dass eine genauere Schranke 
auch einen Anstieg der Komplexität und somit der Laufzeit zur Folge hat. 
Dies liegt daran, dass die bestmögliche Schranke keine Abschätzung mehr 
ist, sondern der korrekte Wert der Lösung. Die eigentliche Suche lässt 
sich somit auf die Ermittlung der Schranke reduzieren. Da die Berechnung 
des Graphabstands NP-vollständig ist, gilt dies somit auch für die 
Berechnung der idealen Schranke.

Ein Ansatz für eine solche Schranke war das Ermitteln eines bipartiten 
Matchings. Die dadurch ermittelte Schranke war jedoch nicht korrekt. Das 
Approximieren des Graphabstandes wäre ein weiterer möglicher Ansatz. 
Lässt sich zu einer möglichen Lösung sagen, wie weit sie höchstens vom 
Optimum ist, dann lässt sich daraus auch eine untere bzw. obere Schranke 
formulieren.


\subsubsection{Aufwandbestimmende Eigenschaften}
Bei der Untersuchung der Testergebnisse des B\&B-Verfahrens zeigte sich, 
dass die Laufzeit von der Kantendichte abhängig sein kann. Dies führte zwar 
nicht zu einer unmittelbaren Verbesserung, jedoch begründet es die Vermutung, 
dass mittels weiterer Eigenschaften sich die Laufzeit besser vorhersagen 
lässt und man die Suche besser optimieren kann.

Mehr Kenntnisse über die Auswirkung verschiedener Grapheigenschaften auf den 
Aufwand könnte auch die Organisation der Suche verbessern. Im Rahmen dieser 
Arbeit gelang eine Verbesserung dadurch, dass die Knoten nach Anzahl ihrer 
Nachbarn sortiert wurden. Eventuell existieren bessere Sortierungen, welche 
die Suche ebenfalls beschleunigen.


\subsection{Evolutionärer Algorithmus}
Dieses lediglich aus Neugierde und Experimentierfreude motivierte Verfahren 
erwies sich als ein hinreichend schnelles Verfahren. Bei kleinen 
Graphenpaaren bedeutete der Mehraufwand für die Verwaltung der Population 
und das Durchlaufen der Generationen noch einen zeitlichen Nachteil 
gegenüber anderen Verfahren. Der Anstieg der Laufzeit ist jedoch deutlich 
geringer. Zusätzlich zur zufriedenstellenden Laufzeit erwies sich auch die 
Qualität der Lösung als hoch. Für kleine bis mittlere Graphenpaare besteht 
sogar eine gute Chance, die beste Lösung zu finden.

Potential für weitere Untersuchungen bieten unter anderem die Wahl der 
Populationsgröße sowie die Abbruchbedingung. Auch die für Rekombination und 
Mutation verwendeten Methoden sind nur erste Ansätze.


\section{Konzept zur praktischen Umsetzung}
Das vorgestellt Konzept befasst sich mit der Umsetzung der LV-Form, der 
Komplexitätsverringerung bei der Suche nach einem ECGM und der Suche nach 
Mustern.

\subsection{Erweiterung}
Das bestehende Konzept lässt sich erweitern, indem man Knoten und Kanten 
beschriftet. Mit einer solchen Beschriftung (engl. Label) ließe sich 
eventuell die Eingabe des Lerners besser nachvollziehen.

So könnte man auf die Vorgabe von Knoten verzichten. Eine solche Vorgabe 
ist einerseits hilfreich, um die Laufzeit beim Vergleichen der Graphen zu 
senken (siehe Abschnitt \ref{sec:VorgabeKnoten}). Andererseits kann sie zur eindeutigen 
Identifikation von Knoten dienen. So unterscheidet sich ein LV-Graph, in dem 
Wölfe Schafe jagen, strukturell nicht von einem LV-Graph, in dem die Schafe 
Jagd auf Wölfe machen.

Statt einer solchen Vorgabe 
vergleicht man die Beschriftung des Lerners mit der Beschriftung der 
Musterlösung. Mittels der so genannten \emph{Levenshtein-Distanz} zwischen 
den Beschriftungen, weist man dann die Knoten der Eingabe Knoten der 
Musterlösung zu. Die Levenshtein-Distanz lässt sich mit quadratischen 
Zeitaufwand ermitteln \cite{wikiD:LevDis}.

Eine weitere Möglichkeit wäre es (wie bereits in Abschnitt \ref{sec:Graphabstand} 
erwähnt), die Kostenfunktion für ein ECGM um die Levenshtein-Distanz zu 
erweitern. Somit würde die Zuordnung indirekt bei der Suche nach einem 
ECGM erfolgen. Dabei gilt es zu beachten, dass es (bei entsprechender 
Wahl der Kosten) billiger sein kann einen Knoten zu löschen und einen 
neuen einzufügen statt die Beschriftung zu ändern. Satz~\ref{alleVausV1enthalten} 
(Abschnitt~\ref{Vorueberlegungen}) wäre dann nicht mehr gültig und somit 
kann auch die Laufzeit zur Ermittlung einer Lösung steigen. Es kann aber 
auch passieren, dass sich bessere Schranken für B\&B formulieren lassen. 
Dies würde dann zu einer besseren Laufzeit führen.

\subsection{Funktionsfähigkeit in der Praxis}
Für alle Teile des Konzepts bleibt jedoch die Frage, ob sie in der Praxis 
funktionieren. So bietet der Ansatz der Knotenfärbung, also das Einteilen 
der Knoten in Gruppen, die Möglichkeit die Anzahl der möglichen 
Permutationen und somit die Laufzeit stark zu reduzieren. Eventuell ist 
dies aber nicht bei allen Modellen sinnvoll.

Auch die Nützlichkeit einer Mustersuche kann bei konkreten Modellen 
gering sein. Bereits die Voraussetzung dafür kann unbrauchbar sein. 
Streng genommen beschreibt der Graphabstand nur den Aufwand für den Umbau 
eines Graphen. Man könnte es als geringsten Reparaturaufwand bezeichnen. 
Es bleibt die Frage, ob sich mittels Graphabstand und Mustersuche auch 
die Fehler des Lerners rekonstruieren lassen. Es gibt einen Unterschied 
zwischen dem Zeigen, wie man es richtig macht, und dem Erklären, was man 
falsch gemacht hat.


\section{Bewerten ohne Musterlösung}
Ein gänzlich anderer Ansatz ist es, keine Musterlösung vorzugeben. 
Stattdessen wird überprüft, ob die Eingabe des Lerners bestimmte 
Eigenschaften erfüllt. Dies bietet sich bei Modellen an, bei denen eine 
kleine strukturelle Änderung zu einer starken semantischen Änderung führt. 
Ebenfalls denkbar wäre ein solcher Ansatz, wenn eine große Zahl an 
möglichen korrekten Lösungen existiert.

Der Ansatz kann jedoch deutlich schwieriger umzusetzen sein. In einem 
hinreichend komplexen Modell können Eigenschaften nur schwer oder gar 
nicht überprüfbar sein. Ist ein Modell beispielsweise Turing-vollständig, 
dann lässt sich das Halteproblem für dieses Modell nicht lösen.

Ebenfalls fraglich ist, ob es hilfreich für den Lerner ist. Man kann dem 
Lerner zwar sagen, was an seiner Lösung fehlt, aber kann man ihm auch sagen, 
was er falsch gemacht hat?

\section{Ausblick}
Es gibt nun mehrere Punkte an denen man das hier vorgestellte weiterentwickeln kann. 
Aus algorithmischer Sicht wäre es interessant, bessere Schranken für das B\&B-Verfahren 
zu finden. So werden beispielsweise in \cite{ApproxGED} eine untere und eine obere Schranke 
vorgestellt. Ebenfalls sinnvoll wäre eine gute Unterscheidungsmöglichkeit, wann B\&B und wann 
ein evolutionärer Algorithmus verwendet werden soll.

Aus E-Learning-Sicht wären zwei Schritte interessant. Zum einen ist zu überprüfen, ob das in 
Kapitel~\ref{chp:Umsetzung} vorgestellte Konzept in der Praxis funktioniert. Zum anderen sollten 
mögliche Muster für die Auswertung der Fehlergraphen gefunden werden. In beiden Fällen können 
die Ergebnisse stark vom Modell abhängen, das dem Lerner vermittelt werden soll. Eventuell gibt 
es jedoch für einige Modelle bereits Erfahrungswerte, auf die zurückgegriffen werden kann.